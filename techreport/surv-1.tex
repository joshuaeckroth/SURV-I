\documentclass{book}
\usepackage{palatino}
\usepackage[pdfborder=0 0 0]{hyperref}
\usepackage{algorithm}
\usepackage{algorithmic}

\setcounter{secnumdepth}{3}
\setcounter{tocdepth}{2}
 
\newcommand{\prog}{{\tt SURV-I}~}

\title{\prog\\Technical Report}
\author{The Ohio State University\\Laboratory for Artificial Intelligence
Research}

\begin{document}
\maketitle
\tableofcontents
%\listoffigures

\chapter{Introduction}
\label{sec:introduction}

\section{Terminology}
\label{sec:terminology}

The entities in \prog are named according to typical surveillance usage. As
indicated in Figure \ref{fig:four-terms}, \emph{detections} are entities
pulled from the camera, which are then discriminated and filtered into a set of
\emph{acquisitions}. These acquisitions are sequenced into \emph{tracks}, which
are in-turn \emph{classified}. The classifications may be utilized in decision
procedures by human or machine.

\begin{figure}[ht]
\label{fig:four-terms}
\begin{center}
\emph{camera} $\rightarrow$ detection $\rightarrow$ acquisition $\rightarrow$
track $\rightarrow$ classification $\rightarrow$ \emph{decision}
\end{center}
\caption{Terminology in \prog}
\end{figure}

\subsection{Detection}
\label{sec:detection}

One or more standard entity detection algorithms are employed in \prog on the
camera video feed to identify entities of interest. These algorithms may be
based on movement, edge detection, heat signatures, or other criteria. In any
case, the entities are packaged as \emph{detections} and provided as input to
the software module that finds acquisitions. A detection typically consists of a
unique identification (an integer based on the video frame when the detection
was identified), a position vector, and possibly shape descriptors, a color
histogram, and other relevant attributes.

\subsection{Acquisition}
\label{sec:acquisition}

Detections are discriminated and filtered into a set of \emph{acquisitions} via
various heuristics that identify noise and other uninteresting phenomena. If a
detection is elevated to an acquisition, then it is worth reasoning about (by
the abducer). If it is noise or uninteresting, however, it is thrown out.

The various heuristics used at this point depend highly on the sensors and their
various attributes. For example, cameras may suffer from sun glares or headlight
reflections, while a tripwire may suffer from minor perturbations. In either
case, these `detections' will be eliminated, while meaningful data is elevated
to an acquisition.\footnote{Future versions of \prog may allow heuristic
parameters to be automatically or manually adjusted in real-time. For
example, a camera may need to filter sun glares in the day and headlight
reflections at night.}

\subsection{Track}
\label{sec:track}

Since acquisitions are transient phenomena, it is the case that each video frame
possesses a entirely new set of acquisitions. A \emph{track}, on the other hand,
is a sequence of acquisitions that are interpreted as single phenomena with
duration.  For example, a collection of acquisitions spanning many frames may be
\emph{tracked} as a single moving entity. Tracks contain an ordered set of
acquisitions that constitute that tracks's \emph{history}.

Tracks may also identify acquisitions that have \emph{split} or \emph{merged}.
If, for example, one frame has several small acquistions in a small space while
the next frame has one large acquisition in the same space, a new track may be
instantiated as consisting of a merge of past acquisitions. On the other hand,
if one large acquisition becomes several small acquisitions, then several new
tracks may be instantiated, consisting of a split of one previous acquisition.
This means that, in the case of a merge, the track's history contains several
acquisitions per frame, while in the case of a split, the track's history
contains one acquisition per frame. In both cases, the frames (or timestamps) of
merges and splits are recorded as attributes of the tracks.

\subsection{Classification}
\label{sec:classification}

Tracks are \emph{classified} based on certain judgments regarding an track's
velocity, trajectory, area or volume, split and merge history, and so on. Such
judgments assign the \emph{class} of the track, such as \emph{vehicle},
\emph{person}, \emph{group of people}, and others.

Classifications are the input to decision processes. Decisions are best made
when information about the \emph{class} of each (interesting) phenomenon in the
surveillance video is known.

\subsection{Hypothesis}
\label{sec:hypothesis}

A hypothesis \emph{explains} an acquisition or track if the acquisition or track
is part of its \emph{explained set}. There are two types of hypotheses, which
are mostly orthogonal in concept: track hypotheses (Section
\ref{sec:track-hypotheses}) and classification hypotheses (Section
\ref{sec:classification-hypotheses}). Both kinds of hypotheses have similar
structures (Section \ref{sec:hypotheses-data-structures-and-methods}).

A hypothesis is \emph{accepted} if, after the abduction process, the hypothesis
has not been tagged \emph{unacceptable} as conflicting with other hypotheses or
implausible. Accepted hypotheses are tagged \emph{acceptable} (Section
\ref{sec:tag}).

\subsubsection{Conflicting hypotheses}
\label{sec:conflicting-hypotheses}

A hypothesis $h_0$ \emph{conflicts} with a different hypothesis $h$ if both
$h_0$ and $h$ explain (see above) the same acquisitions or tracks.

\subsubsection{Plausible, neutral, implausible hypothesis}
\label{sec:plausibility}

Plausibility is computed when a hypotheis is proposed. It is a real number with
the following semantics (with plausibility $p$ and some small $\epsilon$):
\begin{itemize}
    \item $p < -\epsilon$ : implausible
    \item $-\epsilon < p < \epsilon$ : neutral; calculate again when more
        information becomes available
    \item $p > \epsilon$ : plausible
\end{itemize}

\subsubsection{Ambiguous hypotheses}
\label{sec:ambiguous-hypotheses}

A set of hypotheses are \emph{ambiguous} if they explain a common set of
acquisitions or tracks and all of their plausibilities are within a small common
range $\delta$. In such a case, there is no obvious `best' hypothesis among the
set and the abducer cannot discriminate among them (they are, in effect,
`tied').

\subsubsection{Essential hypotheses}

An \emph{essential hypothesis} is the only plausible hypothesis that explains a
given acquisition or track.

\subsubsection{Clear-best hypotheses}

A \emph{clear-best} hypothesis is the single hypothesis among a set that all
explain the same acquisitions or tracks whose plausibility is significantly
greater (beyond some threshold) than all other hypotheses in the same set.

\subsubsection{Weak-best hypotheses}

A \emph{weak-best} hypothesis is similar to a clear-best except the threshold is
lower.

\subsection{World}
\label{sec:world}

The \emph{current world state} is the collection of accepted (see above)
hypotheses, both tracks and classifications.


\chapter{Architecture and algorithms}

\section{Detection}

\subsection{XML output format}

Acquisitions are communicated to the \emph{abducer} via the XML format described
below.

\begin{quote}
{\tt
$<$Frames$>$

$<$Frame time="{\it time in seconds}" number="{\it frame number}"$>$

$<$Acquisition id="{\it acquisition\_id}"
    x="{\it x-coordinate}" y="{\it y-coordinate}"

    \qquad width="{\it width}" height="{\it height}" /$>$

\dots

$<$/Frame$>$

\dots

$<$/Frames$>$
}
\end{quote}

\section{Hypotheses: Data structures and methods}
\label{sec:hypotheses-data-structures-and-methods}

\subsection{Attributes}

\subsubsection{Tag}
\label{sec:tag}

When hypothese are accepted, they are \emph{tagged} as such. A hypothesis may
also be tagged as unacceptable.

\subsection{Inquiring about other hypotheses}

Available accessors:
\begin{center}
\begin{tabular}{p{0.25\linewidth}p{0.5\linewidth}}
{\tt h.type}
    & type of $h$ (such as `NewTrack', `Person', etc)\\
{\tt h.plausibility}
    & plausibility of $h$\\
{\tt h.acquisitions}, {\tt h.tracks}
    & acquisitions or tracks that $h$ explains\\
{\tt h.history}
    & list of past explained acquisitions or tracks (indicating change over
time)\\
\end{tabular}
\end{center}

\subsection{Corroboration}

One hypothesis may \emph{corroborate} one or more others, thereby raising the
others' plausibility by some specified amount.

\subsection{Denials}

Hypotheses may indicated that their own acceptance requires that other
hypotheses must not be accepted. This is known as a \emph{denial}. If such a
hypothesis is accepted, then its denials will be tagged as unacceptable.

\subsubsection{Priors}

All hypotheses have a \emph{prior} plausibility, which is by default 0. This
prior typically plays an additive role in the scoring process: most hypotheses
will compute a score and then add the prior. However, hypotheses are free to
alter the prior or score differently depending on the prior's value.

Priors are used to adjust hypothesis scoring when external, real-world data
becomes available. For example, if we have reasons to believe that a certain
class of hypotheses will be likely, we may set their priors $> 0$.

\section{Abduction}

\subsection{Tracking acquisitions}

\subsubsection{Track hypotheses}
\label{sec:track-hypotheses}

\paragraph{Noise}

(small probability; catch-all if nothing else is better)

\paragraph{New}

\paragraph{Continuing}

\paragraph{Split}

\paragraph{Merge}

\subsection{Classifying tracks}

\subsubsection{Classification hypotheses}
\label{sec:classification-hypotheses}

\paragraph{Confusable}

\begin{itemize}
    \item Bushes
    \item Waving flags
\end{itemize}

\paragraph{Vehicle}

\paragraph{Person}

\paragraph{Group of people}

\subsection{Abduction engine}

The abductive engine is composed of two parts: the Controller and the abduction
algorithm.

\subsubsection{Controller}
\label{sec:controller}

The Controller monitors the abduction process and the system resources.  The
Controller is responsible for starting and halting the abduction process,
updating the World (Section \ref{sec:world}), and recording knowledge in
long-term storage.

\subsubsection{Abduction algorithm}
\label{sec:abduction-algorithm}

\begin{algorithm}[h]
\begin{algorithmic}
\REPEAT
    \STATE generate a conservative set of track hypotheses
    \STATE generate a conservative set of classification hypotheses
    \STATE score track hypotheses
    \STATE score classification hypotheses
    \STATE tag implausible track hypotheses as unacceptable
    \STATE tag implausible classification hypotheses as unacceptable
    \REPEAT[essentials, clear-bests, weak-bests]
        \STATE find track essentials
        \STATE tag essential track hypotheses' conflicts as unacceptable
        \STATE find classification essentials
        \STATE tag essential classification hypotheses' conflicts as
            unacceptable
    \UNTIL{all data is explained or no good choices remain}
\UNTIL{resources exhausted or no more explanatory power in loop}
\end{algorithmic}
\caption{Abduction engine}
\label{alg:abduction-engine}
\end{algorithm}

The inner loop in Algorithm \ref{alg:abduction-engine} iterates until all data
is explained or no good choices remain. With respect to the first condition, all
data is explained when there are no acquisitions nor tracks without an
associated accepted track or classification, respectively. The second
condition stops the loop when the inference engine can make no obviously good
choices among the available unaccepted hypotheses. This means there are no
hypotheses that are weak-best (that is, they are all neutral; implausible
hypotheses have already been tagged as unacceptable), or if there are, that such
hypotheses are ambiguous (Section \ref{sec:ambiguous-hypotheses}).

The Controller (Section \ref{sec:controller}) can stop the engine if resources
are depleted (such as time). The `work' is left as-is and the new sensor data is
acquired. New hypotheses can look at this `work' in their own scoring processes.

\subsection{XML output format}

Tracks are communicated to the \emph{player} via the XML format described in
below.

\begin{quote}
{\tt
...
}
\end{quote}

\section{Player}
\label{sec:player}

\chapter{Evaluation}

\section{Other abductive solutions}

\section{Comparison to abduction machines 1-5}

\section{Experimental setup}

\section{Results}

\chapter{Conclusion}

\end{document}
